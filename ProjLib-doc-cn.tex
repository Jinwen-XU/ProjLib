%! TEX program = xelatex
\documentclass[allowbf]{lebhart}

%%================================
%% 引入工具集
%%================================
\usepackage{PJLtoolkit}
\usepackage{longtable}  % breakable tables
\usepackage{hologo}     % more TeX logo

\UseLanguage{Chinese}

%%================================
%% 排版代码
%%================================
\usepackage{listings}
\definecolor{lightergray}{gray}{0.99}
\lstset{language=[LaTeX]TeX,
    keywordstyle=\color{maintheme},
    basicstyle=\ttfamily,
    commentstyle=\color{forestgreen}\ttfamily,
    stringstyle=\rmfamily,
    showstringspaces=false,
    breaklines=true,
    frame=lines,
    backgroundcolor=\color{lightergray},
    flexiblecolumns=true,
    escapeinside={(*}{*)},
    % numbers=left,
    numberstyle=\scriptsize, stepnumber=1, numbersep=5pt,
    firstnumber=last,
} 
\providecommand{\meta}[1]{$\langle${\normalfont\itshape#1}$\rangle$}
\lstset{moretexcs=%
    {color,CreateTheorem,proofideanameCN,cref,dnf,needgraph,UseLanguage,UseOtherLanguage,AddLanguageSetting,maketitle,address,curraddr,email,keywords,subjclass,thanks,dedicatory,PJLdate,ProjLib
    }
}
\lstnewenvironment{code}% 
{\setkeys{lst}{columns=fullflexible,keepspaces=true}}{}

%%================================
%% remind
%%================================
\newenvironment{remind}[1][提示]{%
    \begin{tcolorbox}[breakable,
        enhanced,
        width = \textwidth,
        colback = white, colbacktitle = paper,
        colframe = gray!50, boxrule=0.2mm,
        coltitle = black,
        fonttitle = \sffamily,
        attach boxed title to top left = {yshift=-\tcboxedtitleheight/2,  xshift=\tcboxedtitlewidth/4},
        boxed title style = {boxrule=0pt, colframe=paper},
        before skip = 0.3cm,
        after skip = 0.3cm,
        top = 3mm,
        bottom = 3mm,
        title={\sffamily #1}]%
}{\end{tcolorbox}}

%%================================
%% 名称
%%================================
\providecommand{\PJLtoolkit}{\textsf{PJLtoolkit}}
\providecommand{\PJLamssim}{\textsf{PJLamssim}}
\providecommand{\PJLauthor}{\textsf{PJLauthor}}
\providecommand{\PJLdatestr}{\textsf{PJLdate}}
\providecommand{\PJLdraft}{\textsf{PJLdraft}}
\providecommand{\PJLlang}{\textsf{PJLlang}}
\providecommand{\PJLlogo}{\textsf{PJLlogo}}
\providecommand{\PJLpaper}{\textsf{PJLpaper}}
\providecommand{\PJLthm}{\textsf{PJLthm}}

%%================================
%% 正文
%%================================
\begin{document}

\title{\ProjLib{} 工具集\\[.3\baselineskip]\normalsize 使用指南}
\author{许锦文}
\thanks{对应版本.~~\texttt{\ProjLib{}~2021/05/20}}
\date{2021年5月,北京}

\maketitle

\begin{abstract}
    \ProjLib{} 工具集之设计目的为简化 \LaTeX{} 文档撰写前的准备工作。只需要加载 \PJLtoolkit{},多语言设置就已准备就绪,定理类环境已被设置好可供直接使用,并且引入了一系列辅助功能。
\end{abstract}

% \tableofcontents

\section{PJLtoolkit}

\subsection{如何加载}

只需要在导言部分加入这一行即可:

\begin{code}
  \usepackage{PJLtoolkit}
\end{code}

\begin{remind}[注意事项]
    由于其内部使用了 \textsf{cleveref},\PJLtoolkit{} 需要放在 \textsf{varioref}、\textsf{hyperref} 的后面。
\end{remind}
\medskip

\subsection{选项}

\PJLtoolkit{} 提供下列选项以供选择:
\begin{itemize}
    \item \texttt{draft} 或 \texttt{fast}
        \begin{itemize}
            \item 快速模式。功能会适当缩减,但能够提高编译速度,建议在撰写阶段使用。
        \end{itemize}
    \item \texttt{palatino}、\texttt{times}、\texttt{garamond}、\texttt{biolinum}
        \begin{itemize}
            \item 字体选项。顾名思义,会加载相应名称的字体。
        \end{itemize}
    \item \texttt{author}
        \begin{itemize}
            \item 加载 \PJLauthor{}。关于其详细功能,请参阅有关这一宏包的小节。
        \end{itemize}
    \item \texttt{amssim}
        \begin{itemize}
            \item 加载 \PJLamssim{}。关于其详细功能,请参阅有关这一宏包的小节。
        \end{itemize}
    \item \texttt{nothms}、\texttt{regionalref}
        \begin{itemize}
            \item 来自 \PJLthm{} 的选项,详细信息请参阅有关这一宏包的小节。
        \end{itemize}
\end{itemize}
另外,还有一些组件的参数可以作为文档类的全局参数传递,例如 \PJLpaper{} 的 \texttt{paperstyle} 和 \texttt{preview} 等。详细信息可以参阅对应的小节。

\section{具体组件}

\subsection{PJLamssim}

\PJLamssim{} 用于模拟 \textsf{amsart} 文档类的部分功能,包括:
\begin{itemize}
    \item \lstinline|\address|、\lstinline|\curraddr|、\lstinline|\email|、\lstinline|\dedicatory| 命令 (前三者由 \PJLauthor{} 提供)
    \item \lstinline|\keywords| 命令
    \item \lstinline|\subjclass| 命令
    \item \lstinline|\thanks| 可以写在 \lstinline|\author| 之外
    \item \lstinline|abstract| 环境可以放在 \lstinline|\maketitle| 的前面
\end{itemize}

这些功能只在标准文档类中启用。在 \AmS{} 文档类中,\PJLamssim{} 不会起到任何效果。

\subsection{PJLauthor}

\PJLauthor{} 提供了 \lstinline|\address|、\lstinline|\curraddr|、\lstinline|\email| 命令,并且允许输入多组用户信息。标准的输入方式是这样的:
\begin{code}
  \author{(*\meta{author 1}*)}
  \address{(*\meta{address 1}*)}
  \email{(*\meta{email 1}*)}
  \author{(*\meta{author 2}*)}
  \address{(*\meta{address 2}*)}
  \email{(*\meta{email 2}*)}
  ...
\end{code}
其中 \lstinline|\address|、\lstinline|\curraddr|、\lstinline|\email| 的相互顺序是不重要的。

\subsection{PJLdate}

\PJLdatestr{} 提供了 \lstinline|\PJLdate|\meta{yyyy-mm-dd} 命令,以将 \meta{yyyy-mm-dd} 转换为对应语言的日期格式显示。例如,在当前的中文语境下,\lstinline|\PJLdate{2022-04-01}| 会被显示为:\PJLdate{2022-04-01}。

关于如何选定语言,请参阅关于 \PJLlang{} 的小节。

\subsection{PJLdraft}

\PJLdraft{} 提供了下列命令:
\begin{itemize}
    \item \lstinline|\dnf| 或 \lstinline|\dnf<...>|。效果为:\dnf~或 \dnf<...>。\\其提示文字与当前语言相对应,例如,在法语模式下将会显示为 \UseOtherLanguage{French}{\dnf}。
    \item \lstinline|\needgraph| 或 \lstinline|\needgraph<...>|。效果为:\needgraph~或 \needgraph<...>其提示文字与当前语言相对应,例如,在法语模式下将会显示为 \UseOtherLanguage{French}{\needgraph}
\end{itemize}

关于如何选定语言,请参阅关于 \PJLlang{} 的小节。

\subsection{PJLlang}

\PJLlang{} 提供了多语言支持,包括简体中文、繁体中文、英文、法文、德文、日文、俄文 (其中中文、日文、俄文需要相应的 \TeX{} 引擎与字体支持)。可以通过下列命令来选定语言:
\begin{itemize}
    \item \lstinline|\UseLanguage{|\meta{language name}\lstinline|}|,用于指定语言,在其后将使用对应的语言设定。
    \begin{itemize}
        \item 既可以用于导言部分,也可以用于正文部分。在不指定语言时,默认选定 “English”。
    \end{itemize}
    \item \lstinline|\UseOtherLanguage{|\meta{language name}\lstinline|}{|\meta{content}\lstinline|}|,用指定的语言的设定排版 \meta{content}。
    \begin{itemize}
        \item 相比较 \lstinline|\UseLanguage|,它不会对行距进行修改,因此中西文字混排时能够保持行距稳定。
    \end{itemize}
\end{itemize}

\meta{language name} 有下列选择:
\begin{itemize}
    \item 简体中文:\texttt{Chinese}、\texttt{chinese}、\texttt{SChinese}、\texttt{schinese}、\texttt{SimplifiedChinese} 或 \texttt{simplifiedchinese}
    \item 繁体中文:\texttt{TChinese}、\texttt{tchinese}、\texttt{TraditionalChinese} 或 \texttt{traditionalchinese}
    \item 英文:\texttt{English} 或 \texttt{english}
    \item 法文:\texttt{French} 或 \texttt{french}
    \item 德文:\texttt{German}、\texttt{german} 或 \texttt{ngerman}
    \item 日文:\texttt{Japanese} 或 \texttt{japanese}
    \item 俄文:\texttt{Russian} 或 \texttt{russian}
\end{itemize}

另外,还可以通过下面的方式来填加相应语言的设置:
\begin{itemize}
    \item \lstinline|\AddLanguageSetting{|\meta{settings}\lstinline|}|
    \begin{itemize}
        \item 向所有支持的语言增加设置 \meta{settings}。
    \end{itemize}
    \item \lstinline|\AddLanguageSetting(|\meta{language name}\lstinline|){|\meta{settings}\lstinline|}|
    \begin{itemize}
        \item 向指定的语言 \meta{language name} 增加设置 \meta{settings}。
    \end{itemize}
\end{itemize}
例如,\lstinline|\AddLanguageSetting(German){\color{orange}}| 可以让所有德语以橙色显示(当然,还需要再加上 \lstinline|\AddLanguageSetting{\color{black}}| 来修正其他语言的颜色)。


\subsection{PJLlogo}

\PJLlogo{} 提供了 \lstinline|\ProjLib| 命令用于绘制 Logo,效果为:\ProjLib{}。它与普通的文字指令效果类似,可以用于不同的字号:
% \begin{itemize}
%     \item \lstinline|\tiny|:\quad {\tiny\ProjLib}
%     \item \lstinline|\scriptsize|:\quad {\scriptsize\ProjLib}
%     \item \lstinline|\footnotesize|:\quad {\footnotesize\ProjLib}
%     \item \lstinline|\normalsize|:\quad {\normalsize\ProjLib}
%     \item \lstinline|\large|:\quad {\large\ProjLib}
%     \item \lstinline|\Large|:\quad {\Large\ProjLib}
%     \item \lstinline|\LARGE|:\quad {\LARGE\ProjLib}
%     \item \lstinline|\huge|:\quad {\huge\ProjLib}
%     \item \lstinline|\Huge|:\quad {\Huge\ProjLib}
% \end{itemize}
% \vspace{-0.5\baselineskip}
\begin{longtable}{ll}
    \lstinline|\tiny|:& {\tiny\ProjLib}\\
    \lstinline|\scriptsize|:& {\scriptsize\ProjLib}\\
    \lstinline|\footnotesize|:& {\footnotesize\ProjLib}\\
    \lstinline|\normalsize|:& {\normalsize\ProjLib}\\
    \lstinline|\large|:& {\large\ProjLib}\\
    \lstinline|\Large|:& {\Large\ProjLib}\\
    \lstinline|\LARGE|:& {\LARGE\ProjLib}\\
    \lstinline|\huge|:& {\huge\ProjLib}\\
    \lstinline|\Huge|:& {\Huge\ProjLib}
\end{longtable}

\subsection{PJLpaper}

\PJLpaper{} 主要用于调节纸张颜色。它支持下列选项:

\begin{itemize}
    \item \texttt{paperstyle = \meta{paper style name}}
        \begin{itemize}
            \item 设定纸张色彩样式。\meta{paper style name} 可供选择的选项有:\texttt{yellow}、\texttt{dark} 与 \texttt{nord}。
        \end{itemize}
    \item \texttt{yellowpaper}、\texttt{darkpaper}、\texttt{nordpaper}
        \begin{itemize}
            \item 设定纸张色彩样式。效果与相应名称的 \texttt{paperstyle} 相同。
        \end{itemize}
    \item \texttt{preview}
        \begin{itemize}
            \item 预览模式,将会把 pdf 文件的白边去掉以方便阅读。
        \end{itemize}
\end{itemize}

为了使用的方便,建议把这些选项作为文档类的全局参数,这样对于文档的纸张设定一目了然。

\subsection{PJLthm}

\PJLthm{} 提供定理类环境的设置。它支持下列选项:
\begin{itemize}
    \item \texttt{nothms}
    \begin{itemize}
        \item 不设定定理类环境。如果你希望使用自己的定理样式,可以使用这一选项。
    \end{itemize}
    \item \texttt{regionalref}
    \begin{itemize}
        \item 在智能引用时,定理类环境的名称随当前语言而变化(默认情况下,引用时会始终采用定理类环境所处语境下的名称;例如,在英文语境中书写的定理,即使稍后在中文语境下引用时,仍将显示为 Theorem)。在 \texttt{fast} 模式下,这一选项默认启用。
    \end{itemize}
\end{itemize}

预设的定理类环境包括:\texttt{assumption}、\texttt{axiom}、\texttt{conjecture}、\texttt{convention}、\texttt{corollary}、\texttt{definition}、\texttt{definition-proposition}、\texttt{definition-theorem}、\texttt{example}、\texttt{exercise}、\texttt{fact}、\texttt{hypothesis}、\texttt{lemma}、\texttt{notation}、\texttt{problem}、\texttt{property}、\texttt{proposition}、\texttt{question}、\texttt{remark}、\texttt{theorem},以及相应的带有星号 \lstinline|*| 的无编号版本。这些定理类环境在显示时会依据当前语言而相应变化,例如在中文模式下 \texttt{theorem} 会显示为“定理”,而在英文模式下则会显示为“Theorem”。关于如何选定语言,请参阅关于 \PJLlang{} 的小节。

在引用定理类环境时,建议使用智能引用 \lstinline|\cref{|\meta{label}\lstinline|}|。这样就不必每次都写上相应环境的名称了。

\medskip
若需要定义新的定理类环境,首先要定义这个环境在所用语言下的名称 \lstinline|\|\meta{name of environment}\meta{language abbr},其中 \meta{language abbr} 是语言的简写,分别为:
\begin{longtable}{ll|ll}
    \texttt{CN} & 简体中文  & \texttt{DE} & 德文\\
    \texttt{TC} & 繁体中文  & \texttt{JP} & 日文\\
    \texttt{EN} & 英文  & \texttt{RU} & 俄文\\
    \texttt{FR} & 法文  & &\\
\end{longtable}

\begin{remind}
    如果要定义名称后带有星号 \lstinline|*| 的环境,那么在上面的 \meta{name of environment} 中不用写星号。
\end{remind}

然后用下面五种方式之一定义这一环境:
\begin{itemize}
    \item \lstinline|\CreateTheorem*{|\meta{name of environment}\lstinline|}|
        \begin{itemize}
            \item 定义不编号的环境 \meta{name of environment}
        \end{itemize}
    \item \lstinline|\CreateTheorem{|\meta{name of environment}\lstinline|}|
        \begin{itemize}
            \item 定义编号环境 \meta{name of environment},按顺序编号
        \end{itemize}
    \item \lstinline|\CreateTheorem{|\meta{name of environment}\lstinline|}[|\meta{numbered like}\lstinline|]|
        \begin{itemize}
            \item 定义编号环境 \meta{name of environment},与 \meta{numbered like} 计数器共用编号
        \end{itemize}
    \item \lstinline|\CreateTheorem{|\meta{name of environment}\lstinline|}<|\meta{numbered within}\lstinline|>|
        \begin{itemize}
            \item 定义编号环境 \meta{name of environment},在 \meta{numbered within} 计数器内编号
        \end{itemize}
    \item \lstinline|\CreateTheorem{|\meta{name of environment}\lstinline|}(|\meta{existed environment}\lstinline|)|\\
    \lstinline|\CreateTheorem*{|\meta{name of environment}\lstinline|}(|\meta{existed environment}\lstinline|)|
        \begin{itemize}
            \item 将 \meta{name of environment} 与 \meta{existed environment} 或 \meta{existed environment}\lstinline|*| 等同。
            \item 这种方式通常在两种情况下比较有用:
                \begin{enumerate}
                    \item 希望定义更简洁的名称。例如,使用 \lstinline|\CreateTheorem{thm}(theorem)|,便可以直接用名称 \texttt{thm} 来撰写定理。
                    \item 希望去除某些环境的编号。例如,使用 \lstinline|\CreateTheorem{remark}(remark*)|,便可以去除 \texttt{remark} 环境的编号。
                \end{enumerate}
        \end{itemize}
\end{itemize}

\begin{remind}
    其内部使用了 \textsf{amsthm},因此传统的 \texttt{theoremstyle} 对其也是适用的,只需在相关定义前标明即可。
\end{remind}

\def\proofideanameCN{思路}
\CreateTheorem*{proofidea*}
\CreateTheorem{proofidea}<subsection>

\bigskip
下面提供一个例子。这三行代码:
\begin{code}
  \def\proofideanameCN{思路}
  \CreateTheorem*{proofidea*}
  \CreateTheorem{proofidea}<subsection>
\end{code}
可以分别定义不编号的环境 \lstinline|proofidea*| 和编号的环境 \lstinline|proofidea| (在 subsection 内编号),它们支持在简体中文语境中使用,效果如下所示 (具体样式与所在的文档类有关) :

\begin{proofidea*}
    \lstinline|proofidea*| 环境。
\end{proofidea*}

\begin{proofidea}
    \lstinline|proofidea| 环境。
\end{proofidea}


\medskip
\section{目前存在的问题}

\begin{itemize}[itemsep=.6em]
    \item \PJLauthor{} 仍然处于初步阶段,在很多方面还远远比不上相对成熟的 \textsf{authblk}。
    \item \PJLlang{}:针对 \textsf{polyglossia} 的设置仍然存在许多问题,因此现在主要功能都是通过 \textsf{babel} 实现的。
    \item \PJLpaper{} 的 \texttt{preview} 功能主要是通过 \textsf{geometry} 宏包实现的,因此在 \textsc{\textsf{Koma}} 文档类中效果不好。
    \item \PJLthm{} 对于定理类环境的编号与样式设定目前还无法由用户更改。
    \item \PJLthm{}:智能引用针对所有 \PJLlang{} 已支持语言的本地化尚不完整,主要是中文、日文与俄文。
    \item 错误处理功能不完善,在出现一些问题时没有相应的错误提示。
    \item 代码中仍有许多可优化之处,有些部分耗时过长,特别是 \PJLthm{} 对定理类环境的定义。
\end{itemize}


\clearpage
\section{使用示例}

\subsection{标准文档类}

在标准文档类中,通常只需简要设置页面尺寸、超链接,再载入 \PJLtoolkit{},即可直接开始写作。下面是一段完整的示例。

\begin{code}
\documentclass{article}
\usepackage[a4paper,margin=.75in]{geometry}
\usepackage[hidelinks]{hyperref}
\usepackage[palatino]{PJLtoolkit} % Load the toolkit and use font Palatino

\UseLanguage{French} % Use French from here

\begin{document}

\title{Le Titre}
\author{Auteur}
\date{\PJLdate{2022-04-01}}

\maketitle

\begin{abstract}
    Ceci est un résumé. \dnf<Plus de contenu est nécessaire.>
\end{abstract}

\section{Un théorème}

%% Theorem-like environments can be used directly
\begin{theorem}\label{thm:abc}
    Ceci est un théorème.
\end{theorem}

Référence du théorème: \cref{thm:abc} % It is recommended to use clever reference

\end{document}
\end{code}

\bigskip
如果使用了 \PJLamssim{},那么文章中就可以采用 \AmS{} 文档类的写法 (当然,此时原始的写法也是成立的,因此始终添加 \texttt{amssim} 这一参数通常是没有问题的)。此时上文引入 \PJLtoolkit{} 的那一行应该写为:
\begin{code}
\usepackage[amssim,palatino]{PJLtoolkit}
\end{code}

\clearpage
\subsection{\texorpdfstring{\AmS{}}{AMS} 文档类}

在 \AmS{} 文档类中,通常只需简要设置页面尺寸、超链接,再载入 \PJLtoolkit{},即可直接开始写作。下面是一段完整的示例。
\begin{code}
\documentclass{amsart}
\usepackage[a4paper,margin=.75in]{geometry}
\usepackage[hidelinks]{hyperref}
\usepackage[palatino]{PJLtoolkit} % Load the toolkit and use font Palatino

\UseLanguage{French} % Use French from here

\begin{document}

\title{Le Titre}
\author{Auteur 1}
\address{Adresse 1}
\email{\href{Courriel 1}{Courriel 1}}
\author{Auteur 1}
\address{Adresse 1}
\email{\href{Courriel 2}{Courriel 2}}
\date{\PJLdate{2022-04-01}}
\subjclass{*****}
\keywords{...}

\begin{abstract}
    Ceci est un résumé. \dnf<Plus de contenu est nécessaire.>
\end{abstract}

\maketitle

\section{Première section}

%% Theorem-like environments can be used directly
\begin{theorem}\label{thm:abc}
    Ceci est un théorème.
\end{theorem}

Référence du théorème: \cref{thm:abc} % It is recommended to use clever reference

\end{document}
\end{code}


\end{document}
