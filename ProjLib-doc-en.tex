%! TEX program = xelatex
\documentclass[English,French,allowbf,regionalref,puretext]{lebhart}

%%================================
%% Import toolkit
%%================================
\usepackage{ProjLib}
\usepackage{longtable}  % breakable tables
\usepackage{hologo}     % more TeX logo

%%================================
%% For typestting code
%%================================
\usepackage{listings}
\definecolor{lightergray}{gray}{0.99}
\lstset{language=[LaTeX]TeX,
    keywordstyle=\color{maintheme},
    basicstyle=\ttfamily,
    commentstyle=\color{forestgreen}\ttfamily,
    stringstyle=\rmfamily,
    showstringspaces=false,
    breaklines=true,
    frame=lines,
    backgroundcolor=\color{lightergray},
    flexiblecolumns=true,
    escapeinside={(*}{*)},
    % numbers=left,
    numberstyle=\scriptsize, stepnumber=1, numbersep=5pt,
    firstnumber=last,
} 
\providecommand{\meta}[1]{$\langle${\normalfont\itshape#1}$\rangle$}
\lstset{moretexcs=%
    {color,NameTheorem,CreateTheorem,proofideanameEN,cref,dnf,needgraph,UseLanguage,UseOtherLanguage,AddLanguageSetting,maketitle,address,curraddr,email,keywords,subjclass,thanks,dedicatory,PLdate,PJLdate,ProjLib,
    mathfrak,mf,frak,mfA,mathbb,bb,N,Z,Q,R,C,F,A,mathcal,mc,cal,mathscr,ms,scr,
    }
}
\lstnewenvironment{code}% 
{\setkeys{lst}{columns=fullflexible,keepspaces=true}}{}

%%================================
%% tip
%%================================
\newenvironment{tip}[1][Tip]{%
    \begin{tcolorbox}[breakable,
        enhanced,
        width = \textwidth,
        colback = paper, colbacktitle = paper,
        colframe = gray!50, boxrule=0.2mm,
        coltitle = black,
        fonttitle = \sffamily,
        attach boxed title to top left = {yshift=-\tcboxedtitleheight/2, xshift=.5cm},
        boxed title style = {boxrule=0pt, colframe=paper},
        before skip = 0.3cm,
        after skip = 0.3cm,
        top = 3mm,
        bottom = 3mm,
        title={\scshape\sffamily #1}]%
}{\end{tcolorbox}}

%%================================
%% 名称
%%================================
\providecommand{\ProjLibPackage}{\textsf{ProjLib}}
\providecommand{\PJLtoolkit}{\textsf{PJLtoolkit}}
\providecommand{\PJLamssim}{\textsf{PJLamssim}}
\providecommand{\PJLauthor}{\textsf{PJLauthor}}
\providecommand{\PJLdatePackage}{\textsf{PJLdate}}
\providecommand{\PJLdraft}{\textsf{PJLdraft}}
\providecommand{\PJLlang}{\textsf{PJLlang}}
\providecommand{\PJLlogo}{\textsf{PJLlogo}}
\providecommand{\PJLmath}{\textsf{PJLmath}}
\providecommand{\PJLpaper}{\textsf{PJLpaper}}
\providecommand{\PJLthm}{\textsf{PJLthm}}

%%================================
%% 正文
%%================================
\begin{document}

\title{\ProjLib{}~\,Toolkit\\[.3\baselineskip]\normalsize User Manual}
\author{Jinwen Xu}
\thanks{Corresponding to: \texttt{\ProjLib{}~2021/06/28}}
\date{June 2021, Beijing}

\maketitle

\begin{abstract}
    The \ProjLib{} toolkit is designed to simplify the preparation before writing \LaTeX{} documents. With \ProjLibPackage{} loaded, you no longer need to set up the theorem-like environments nor configure the appropriate multilingual settings. Additionally, a series of auxiliary functionalities are introduced.
\end{abstract}

% \tableofcontents

\section{The main package}

\subsection{How to load it}

Just add the following line to your preamble:

\begin{code}
  \usepackage{ProjLib}
\end{code}

\begin{tip}[Attention]
    Since \textsf{cleveref} is used internally, \ProjLibPackage{} needs to be placed after \textsf{varioref} and \textsf{hyperref}.
\end{tip}
\medskip

\subsection{Options}

\ProjLibPackage{} offers the following options:
\begin{itemize}
    \item \texttt{draft} or \texttt{fast}
        \begin{itemize}
            \item Fast mode. The functionality will be appropriately reduced to get faster compilation speed, recommended to use during the writing phase.
        \end{itemize}
    \item \texttt{palatino}, \texttt{times}, \texttt{garamond}, \texttt{biolinum} ~$|$~ \texttt{useosf}
        \begin{itemize}
            \item Font options. As the name suggest, font with corresponding name will be loaded. 
            \item The \texttt{useosf} option is used to enable the old-style figures.
        \end{itemize}
    \item \texttt{nothms}, \texttt{nothmnum}, \texttt{regionalref}
        \begin{itemize}
            \item Options from \PJLthm{}, please refer to the section on this package for details.
        \end{itemize}
    \item \texttt{author}
        \begin{itemize}
            \item Load \PJLauthor{}. For more information about its functionality, see the section on this package.
        \end{itemize}
    \item \texttt{amssim}
        \begin{itemize}
            \item Load \PJLamssim{}. For more information about its functionality, see the section on this package.
        \end{itemize}
\end{itemize}
In addition, there are also some options of the components that should be passed as global options of your document class, such as the language options \texttt{EN} / \texttt{english} / \texttt{English}, \texttt{FR} / \texttt{french} / \texttt{French} etc.\,of \PJLlang{}, \\and \texttt{paperstyle}, \texttt{preview} of \PJLpaper{}. For more information, please refer to the corresponding sections.

\section{The components}

\subsection{PJLamssim}

\PJLamssim{} is used to simulate some features of the \textsf{amsart} class in a standard class, including:
\begin{itemize}
    \item \lstinline|\address|, \lstinline|\curraddr|, \lstinline|\email| and \lstinline|\dedicatory| macro(the first three are provided by \PJLauthor{})
    \item \lstinline|\keywords| macro
    \item \lstinline|\subjclass| macro
    \item \lstinline|\thanks| can be written outside \lstinline|\author|
    \item The \lstinline|abstract| environment can be placed before \lstinline|\maketitle|
\end{itemize}

These modifications would only take place in standard classes. In the \AmS{} classes, \PJLamssim{} does not have any effect.

\subsection{PJLauthor}

\PJLauthor{} offers \lstinline|\address|, \lstinline|\curraddr| and \lstinline|\email|, and allows you to enter multiple groups of author information. The standard usage is like this:
\begin{code}
  \author{(*\meta{author 1}*)}
  \address{(*\meta{address 1}*)}
  \email{(*\meta{email 1}*)}
  \author{(*\meta{author 2}*)}
  \address{(*\meta{address 2}*)}
  \email{(*\meta{email 2}*)}
  ...
\end{code}
The mutual order of \lstinline|\address|, \lstinline|\curraddr| and \lstinline|\email| is not important.

\subsection{PJLdate}

\PJLdatePackage{} offers the \lstinline|\PLdate|\meta{yyyy-mm-dd} (or \lstinline|\PJLdate|\meta{yyyy-mm-dd}) macro to convert \meta{yyyy-mm-dd} into the date format of the currently selected language. For example, in current English context, \lstinline|\PLdate{2022-04-01}| would become ``\PLdate{2022-04-01}'', while in French context as ``\UseOtherLanguage{French}{\PLdate{2022-04-01}}''.

For details on how to select a language, please refer to the section on \PJLlang{}.

\subsection{PJLdraft}

\PJLdraft{} offers the following macros:
\begin{itemize}
    \item \lstinline|\dnf| or \lstinline|\dnf<...>|. The effect is: \dnf~ or \dnf<...>. \\The prompt text changes according to the current language. For example, it will be displayed as \UseOtherLanguage{French}{\dnf} in French mode.
    \item \lstinline|\needgraph| or \lstinline|\needgraph<...>|. The effect is: \needgraph or \needgraph<...>The prompt text changes according to the current language. For example, in French mode, it will be displayed as \UseOtherLanguage{French}{\needgraph}
\end{itemize}

For details on how to select a language, please refer to the section on \PJLlang{}.

\subsection{PJLlang}

\PJLlang{} offers multi-language support, including simplified Chinese, traditional Chinese, English, French, German, Japanese, and Russian (among them, Chinese, Japanese, and Russian require appropriate \TeX{} engines and fonts to support). 

\medskip
\PJLlang{} provides language options. The names of these options have three types, which are abbreviations (such as \texttt{EN}), lowercase (such as \texttt{english}), and capital letters (such as \texttt{English}). For the option names of a specific language, please refer to \meta{language name} below. Among them, the first specified language \meta{first language} will be used as the default language, which is equivalent to specifying \lstinline|\UseLanguage{|\meta{first language}\lstinline|}| at the beginning of your document.

\begin{tip}
    It is recommended to use these language options and pass them as global options. In this way, only the specified language is set, thus saving the \TeX{} memory and significantly improving the compilation speed.
\end{tip}

\medskip
The language can be selected by the following macros:

\begin{itemize}
    \item \lstinline|\UseLanguage{|\meta{language name}\lstinline|}| is used to specify the language. The corresponding setting of the language will be applied after it. It can be used either in the preamble or in the main body. When no language is specified, ``English" is selected by default.
    \item \lstinline|\UseOtherLanguage{|\meta{language name}\lstinline|}{|\meta{content}\lstinline|}|, which uses the specified language settings to typeset \meta{content}. Compared with \lstinline|\UseLanguage|, it will not modify the line spacing, so line spacing would remain stable when CJK and Western texts are mixed.
\end{itemize}

\medskip
\meta{language name} can be (it is not case sensitive, for example, \texttt{French} and \texttt{french} have the same effect):
\begin{itemize}
    \item Simplified Chinese: \texttt{CN}, \texttt{Chinese}, \texttt{SChinese} or \texttt{SimplifiedChinese}
    \item Traditional Chinese: \texttt{TC}, \texttt{TChinese} or \texttt{TraditionalChinese}
    \item English: \texttt{EN} or \texttt{English}
    \item French: \texttt{FR} or \texttt{French}
    \item German: \texttt{DE}, \texttt{German} or \texttt{ngerman}
    \item Italian: \texttt{IT} or \texttt{Italian}
    \item Portuguese: \texttt{PT} or \texttt{Portuguese}
    \item Portuguese (Brazilian): \texttt{BR} or \texttt{Brazilian}
    \item Spanish: \texttt{ES} or \texttt{Spanish}
    \item Japanese: \texttt{JP} or \texttt{Japanese}
    \item Russian: \texttt{RU} or \texttt{Russian}
\end{itemize}

\medskip
In addition, you can also add new settings to selected language:
\begin{itemize}
    \item \lstinline|\AddLanguageSetting{|\meta{settings}\lstinline|}|
    \begin{itemize}
        \item Add \meta{settings} to all supported languages.
    \end{itemize}
    \item \lstinline|\AddLanguageSetting(|\meta{language name}\lstinline|){|\meta{settings}\lstinline|}|
    \begin{itemize}
        \item Add \meta{settings} to the selected language \meta{language name}.
    \end{itemize}
\end{itemize}
For example, \lstinline|\AddLanguageSetting(German){\color{orange}}| can make all German text displayed in orange (of course, one then need to add \lstinline|\AddLanguageSetting{\color{black}}| in order to correct the color of the text in other languages).

\clearpage
\subsection{PJLlogo}

\PJLlogo{} offers the \lstinline|\ProjLib| macro to draw the logo, which looks like \ProjLib{}. It is similar to ordinary text macros and can be used with different font size macros:
% \begin{itemize}
%     \item \lstinline|\tiny|:\quad {\tiny\ProjLib}
%     \item \lstinline|\scriptsize|:\quad {\scriptsize\ProjLib}
%     \item \lstinline|\footnotesize|:\quad {\footnotesize\ProjLib}
%     \item \lstinline|\normalsize|:\quad {\normalsize\ProjLib}
%     \item \lstinline|\large|:\quad {\large\ProjLib}
%     \item \lstinline|\Large|:\quad {\Large\ProjLib}
%     \item \lstinline|\LARGE|:\quad {\LARGE\ProjLib}
%     \item \lstinline|\huge|:\quad {\huge\ProjLib}
%     \item \lstinline|\Huge|:\quad {\Huge\ProjLib}
% \end{itemize}
% \bigskip

% \begin{longtable}{ll}
\begin{center}
    \begin{tabular}{ll} 
        \lstinline|\tiny|:& {\tiny\ProjLib}\\
        \lstinline|\scriptsize|:& {\scriptsize\ProjLib}\\
        \lstinline|\footnotesize|:& {\footnotesize\ProjLib}\\
        \lstinline|\normalsize|:& {\normalsize\ProjLib}\\
        \lstinline|\large|:& {\large\ProjLib}\\
        \lstinline|\Large|:& {\Large\ProjLib}\\
        \lstinline|\LARGE|:& {\LARGE\ProjLib}\\
        \lstinline|\huge|:& {\huge\ProjLib}\\
        \lstinline|\Huge|:& {\Huge\ProjLib}
    \end{tabular}
\end{center}
% \end{longtable}

\subsection{PJLmath}

\PJLmath{} offers the following shortcuts:
\begin{enumerate}[label=\roman*)]
    \item \lstinline|\mathfrak{|$\cdot$\lstinline|}| $\longrightarrow$ \lstinline|\mf|$\cdot$ or \lstinline|\frak|$\cdot$ . For example, \lstinline|\mfA| (or \lstinline|\mf{A}|) has the same effect as \lstinline|\mathfrak{A}|. This works for both upper and lower case, producing:
    \begin{align*}
        \mfa\mfb\mfc\mfd\mfe\mff\mfg\mfh\mfi\mfj\mfk\mfl\mfm\mfn&\mfo\mfp\mfq\mfr\mfs\mft\mfu\mfv\mfw\mfx\mfy\mfz\\
        \mfA\mfB\mfC\mfD\mfE\mfF\mfG\mfH\mfI\mfJ\mfK\mfL\mfM\mfN&\mfO\mfP\mfQ\mfR\mfS\mfT\mfU\mfV\mfW\mfX\mfY\mfZ
    \end{align*}
    \item \lstinline|\mathbb{|$\cdot$\lstinline|}| \( \longrightarrow \) \lstinline|\bb|$\cdot$ . This only works for uppercase alphabet and the number \( 1 \).
    \begin{equation*}
        \bbA\bbB\bbC\bbD\bbE\bbF\bbG\bbH\bbI\bbJ\bbK\bbL\bbM\bbN\bbO\bbP\bbQ\bbR\bbS\bbT\bbU\bbV\bbW\bbX\bbY\bbZ\bb1
    \end{equation*}
    There are also special command for well-known algebraic structures: \lstinline|\N|, \lstinline|\Z|, \lstinline|\Q|, \lstinline|\R|, \lstinline|\C|, \lstinline|\F|, \lstinline|\A|.
    \[
        \N\Z\Q\R\C\F\A
    \]
    \item \lstinline|\mathcal{|$\cdot$\lstinline|}| \( \longrightarrow \) \lstinline|\mc|$\cdot$ or \lstinline|\cal|$\cdot$ . This only works for uppercase alphabet.
    \begin{equation*}
        \mcA\mcB\mcC\mcD\mcE\mcF\mcG\mcH\mcI\mcJ\mcK\mcL\mcM\mcN\mcO\mcP\mcQ\mcR\mcS\mcT\mcU\mcV\mcW\mcX\mcY\mcZ
    \end{equation*}
    \item \lstinline|\mathscr{|$\cdot$\lstinline|}| \( \longrightarrow \) \lstinline|\ms|$\cdot$ or \lstinline|\scr|$\cdot$ . This only works for uppercase alphabet.
    \begin{equation*}
        \msA\msB\msC\msD\msE\msF\msG\msH\msI\msJ\msK\msL\msM\msN\msO\msP\msQ\msR\msS\msT\msU\msV\msW\msX\msY\msZ
    \end{equation*}
\end{enumerate}

% \clearpage
\subsection{PJLpaper}

\PJLpaper{} is mainly used to adjust the paper color. It has the following options:

\vspace{-.3\baselineskip}
\begin{itemize}
    \item \texttt{paperstyle = \meta{paper style name}}
        \begin{itemize}
            \item Set the paper color style. The options available for \meta{paper style name} are: \texttt{yellow}, \texttt{dark} and \texttt{nord}.
        \end{itemize}
    \item \texttt{yellowpaper}, \texttt{darkpaper}, \texttt{nordpaper}
        \begin{itemize}
            \item Same as \texttt{paperstyle} with the corresponding \meta{paper style name} specified.
        \end{itemize}
    \item \texttt{preview}
        \begin{itemize}
            \item Preview mode. Crop the white edges of pdf file for the convenience of reading.
        \end{itemize}
\end{itemize}
\vspace{-.3\baselineskip}

It is recommended to use them as global options of the document class. In this way, the paper settings would be clear at a glance.

\subsection{PJLthm}

\PJLthm{} offers the configuration of theorem-like environments. It has the following option:

\vspace{-.3\baselineskip}
\begin{itemize}
    \item \texttt{nothms}
    \begin{itemize}
        \item Theorem-like environments will not be defined. You may use this option if you wish to apply your own theorem styles.
    \end{itemize}
    \item \texttt{nothmnum}
    \begin{itemize}
        \item Theorem-like environments will not be numbered.
    \end{itemize}
    \item \texttt{regionalref}
    \begin{itemize}
        \item When referencing, name of the theorem-like environment will change with the current language (by default, the name will always remain the same; for example, when referencing a theorem written in the French context, even if one is currently in the English context, it will still be displayed as ``Théorème''). In \texttt{fast} mode, this option is automatically enabled.
    \end{itemize}
\end{itemize}

\medskip
Preset environments include: \texttt{assumption}, \texttt{axiom}, \texttt{conjecture}, \texttt{convention}, \texttt{corollary}, \texttt{definition}, \texttt{definition-proposition}, ~~\texttt{definition-theorem}, ~~\texttt{example}, ~~\texttt{exercise}, ~~\texttt{fact}, ~~\texttt{hypothesis}, ~~\texttt{lemma}, \\\texttt{notation}, \texttt{observation}, \texttt{problem}, \texttt{property}, \texttt{proposition}, \texttt{question}, \texttt{remark}, \texttt{theorem}, and the corresponding unnumbered version with an asterisk \lstinline|*| in the name. The titles will change with the current language. For example, \texttt{theorem} will be displayed as ``Theorem" in English mode and ``Théorème" in French mode. For details on how to select a language, please refer to the section on \PJLlang{}.

\begin{tip}
    When referencing a theorem-like environment, it is recommended to use \lstinline|\cref{|\meta{label}\texttt{\}}. In this way, there is no need to explicitly write down the name of the corresponding environment every time.
\end{tip}

\medskip
% If you need to define a new theorem-like environment, you must first define the name of the environment in the language used: \lstinline|\|\meta{name of environment}\meta{language abbr}, where \meta{language abbr} is the abbreviation of language, which can be:
% % \vspace{-.5\baselineskip}
% \begin{longtable}{ll|ll}
%     \texttt{CN} & Simplified Chinese & \texttt{DE} & German\\
%     \texttt{TC} & Traditional Chinese & \texttt{JP} & Japanese\\
%     \texttt{EN} & English & \texttt{RU} & Russian\\
%     \texttt{FR} & French & &\\
% \end{longtable}

% \begin{tip}
%     If you want to define an environment with an asterisk \lstinline|*| after the name, you don't need to write an asterisk in the \meta{name of environment} above. See the example below.
% \end{tip}
\medskip
If you need to define a new theorem-like environment, you must first define the name of the environment in the language to use:
\begin{itemize}
    \item \lstinline|\NameTheorem[|\meta{language name}\lstinline|]{|\meta{name of environment}\lstinline|}{|\meta{name string}\lstinline|}|
\end{itemize}
For \meta{language name}, please refer to the section on \PJLlang{}. When \meta{language name} is not specified, the name will be set for all supported languages. In addition, environments with or without asterisk share the same name, therefore, \lstinline|\NameTheorem{envname*}{...}| has the same effect as \lstinline|\NameTheorem{envname}{...}| .

\bigskip
And then define this environment in one of following five ways:
\vspace{-.3\baselineskip}
\begin{itemize}
    \item \lstinline|\CreateTheorem*{|\meta{name of environment}\lstinline|}|
        \begin{itemize}
            \item Define an unnumbered environment \meta{name of environment}
        \end{itemize}
    \item \lstinline|\CreateTheorem{|\meta{name of environment}\lstinline|}|
        \begin{itemize}
            \item Define a numbered environment \meta{name of environment}, numbered in order 1,2,3,\dots
        \end{itemize}
    \item \lstinline|\CreateTheorem{|\meta{name of environment}\lstinline|}[|\meta{numbered like}\lstinline|]|
        \begin{itemize}
            \item Define a numbered environment \meta{name of environment}, which shares the counter \meta{numbered like}
        \end{itemize}
    \item \lstinline|\CreateTheorem{|\meta{name of environment}\lstinline|}<|\meta{numbered within}\lstinline|>|
        \begin{itemize}
            \item Define a numbered environment \meta{name of environment}, numbered within the counter \meta{numbered within}
        \end{itemize}
    \clearpage
    \item \lstinline|\CreateTheorem{|\meta{name of environment}\lstinline|}(|\meta{existed environment}\lstinline|)|\\
    \lstinline|\CreateTheorem*{|\meta{name of environment}\lstinline|}(|\meta{existed environment}\lstinline|)|
        \begin{itemize}
            \item Identify \meta{name of environment} with \meta{existed environment} or \meta{existed environment}\lstinline|*|.
            \item This method is usually useful in the following two situations:
                \begin{enumerate}
                    \item To use a more concise name. For example, with \lstinline|\CreateTheorem{thm}(theorem)|, one can then use the name \texttt{thm} to write theorem.
                    \item To remove the numbering of some environments. For example, one can remove the numbering of the \texttt{remark} environment with \lstinline|\CreateTheorem{remark}(remark*)|.
                \end{enumerate}
        \end{itemize}
\end{itemize}

\begin{tip}
    This macro utilizes the feature of \textsf{amsthm} internally, so the traditional \texttt{theoremstyle} is also applicable to it. One only needs declare the style before the relevant definitions.
\end{tip}

% \def\proofideanameEN{Idea}
\NameTheorem[EN]{proofidea}{Idea}
\CreateTheorem*{proofidea*}
\CreateTheorem{proofidea}<subsection>

\bigskip
Here is an example. The following code:
\begin{code}
  \NameTheorem[EN]{proofidea}{Idea}
  \CreateTheorem*{proofidea*}
  \CreateTheorem{proofidea}<subsection>
\end{code}
defines an unnumbered environment \lstinline|proofidea*| and a numbered environment \lstinline|proofidea| (numbered within subsection) respectively. They can be used in English context. The effect is as follows (the actual style is related to the document class):

\begin{proofidea*}
    The \lstinline|proofidea*| environment.
\end{proofidea*}

\begin{proofidea}
    The \lstinline|proofidea| environment.
\end{proofidea}


\medskip
\section{Known issues}

\begin{itemize}[itemsep=.6em]
    \item \PJLauthor{} is still in its preliminary stage, its effect is not as good as the relatively mature \textsf{authblk}.
    \item \PJLlang{}: It is still quite problematic with the configuration of \textsf{polyglossia}, so main features are implemented through \textsf{babel} for now.
    \item \PJLlang{}: There are some problems with the language options. For example, \texttt{chinese} will cause errors with \textsf{babel}. Also, conflicts among multiple options may occur.
    \item \PJLpaper{}: the \texttt{preview} option is mainly implemented with the help of package \textsf{geometry}, so it does not work quite as well in the \textsc{\textsf{Koma}} document classes.
    \item \PJLthm{}: The numbering and theorem-style settings of the theorem-like environments cannot be accessed by the user at present.
    \item \PJLthm{}: The localization of \textsf{cleveref} is not yet complete for all supported languages of \PJLlang{}, especially for Chinese, Japanese and Russian.
    \item Error handling mechanism is incomplete: no corresponding error prompt when some problems occur.
    \item There are still many things that can be optimized in the code. Some takes too long to run, especially the setup of theorem-like environments in \PJLthm{}.
\end{itemize}


\clearpage
\section{Usage example}

\subsection{Standard classes}

In standard classes, one usually only need to configure the page size, hyperlinks and load \ProjLibPackage{} before actually start writing the document. Below is a complete example.

\begin{code}
\documentclass{article}
\usepackage[a4paper,margin=.75in]{geometry}
\usepackage[hidelinks]{hyperref}
\usepackage[palatino]{ProjLib} % Load the toolkit and use font Palatino

\UseLanguage{French} % Use French from here

\begin{document}

\title{Le Titre}
\author{Auteur}
\date{\PJLdate{2022-04-01}}

\maketitle

\begin{abstract}
    Ceci est un résumé. \dnf<Plus de contenu est nécessaire.>
\end{abstract}

\section{Un théorème}

%% Theorem-like environments can be used directly
\begin{theorem}\label{thm:abc}
    Ceci est un théorème.
\end{theorem}

Référence du théorème: \cref{thm:abc} % It is recommended to use clever reference

\end{document}
\end{code}

\bigskip
If \PJLamssim{} is loaded, then one can adopt the \AmS{} writing style in the document (of course, the original way is also valid, so always adding the option \texttt{amssim} usually does not cause problems). This way, the line that introduces \ProjLibPackage{} should be written as:
\begin{code}
\usepackage[amssim,palatino]{ProjLib}
\end{code}

\clearpage
\subsection{The \texorpdfstring{\AmS{}}{AMS} classes}

In \AmS{} classes, one usually only need to configure the page size, hyperlinks and load \ProjLibPackage{} before actually start writing the document. Below is a complete example.
\begin{code}
\documentclass{amsart}
\usepackage[a4paper,margin=.75in]{geometry}
\usepackage[hidelinks]{hyperref}
\usepackage[palatino]{ProjLib} % Load the toolkit and use font Palatino

\UseLanguage{French} % Use French from here

\begin{document}

\title{Le Titre}
\author{Auteur 1}
\address{Adresse 1}
\email{\href{Courriel 1}{Courriel 1}}
\author{Auteur 1}
\address{Adresse 1}
\email{\href{Courriel 2}{Courriel 2}}
\date{\PJLdate{2022-04-01}}
\subjclass{*****}
\keywords{...}

\begin{abstract}
    Ceci est un résumé. \dnf<Plus de contenu est nécessaire.>
\end{abstract}

\maketitle

\section{Première section}

%% Theorem-like environments can be used directly
\begin{theorem}\label{thm:abc}
    Ceci est un théorème.
\end{theorem}

Référence du théorème: \cref{thm:abc} % It is recommended to use clever reference

\end{document}
\end{code}


\end{document}
